\documentclass{report}
\usepackage{graphicx}
\usepackage{color}
\definecolor{ForestGreen}{RGB}{100, 200, 45}
\definecolor{ApBlue}{RGB}{75, 0, 200}

\graphicspath{ {/home/riky/Pictures/Wallpapers/Misc/} }
\author{Oglietti Riccardo \\ Ye, it's a \LaTeX{}}
\title{Prove Sparse}
\date{\today}

\begin{document}
\maketitle
\begin{abstract}
    Il documento parlera' di nulla, con del nulla su lol perche' avevo quell'immagine a portata.
\end{abstract}
    First document. This is a simple example, with no extra
    parameters or pakage included.
    \begin{itemize}
        \item Item 1
        \item Item 2
        \item Item 3
    \end{itemize}
    Questa e' una tabella!
    \begin{center}
        \begin{tabular}{ |c|c|c| }
            \hline
            Num & Num & Num \\ [0.5ex] %lo 0.5 e' lo spazio sotto la riga se non ho capito male
            \hline\hline
            1 & 2 & 3 \\
            \hline
            4 & 5 & 6 \\
            \hline
            7 & 8 & 9 \\
            \hline
        \end{tabular}
    \end{center}
    Altro coso, seguito da un newpage
    \newpage
    \chapter{Zyra}

    \section{Il Pick}
    \begin{figure}[h]
        \centering
        \includegraphics [width=12cm] {Zyra.jpg}
        %\includegraphics [scale=0.3] {Zyra.jpg}
        \caption{actually nice}
        \label{fig:Zyra}
    \end{figure}

    Quella in figura \ref{fig:Zyra} e' il mio main su \underline{lol}, fa \textit{pew pew} con le \textbf{piantine}
    Questa \emph{frase} necessita di enfasi varie, cosi' mi pare un po' \textit{troppo \textbf{moscia}.}

    Zyra ha quattro abilita' fondamentali, come molti atlry campioni,
    \begin{enumerate}
        \item Q, \textbf{Deadly Spines}, fa \emph{cose} con delle spine a 90 gradi rispetto a lei, se per qualche motivo colpisce un bocciolo, quello spara come fosse uno scolaro americano
        \item W, \textbf{Rampant Growth}, pianta boccioli strani, se vengono colpiti dalla Q o dalla E diventano piante.
        \item E, \textbf{Grasping Roots}, lacia una striscia di piante davanri a se che rootta gli avversari, e fa \textbf{male}.
        \item R, \textbf{Strangle Thorns}, fa male, da airborne ai nemici, e le piante li intorno diventano dei faking \textcolor{ForestGreen}{mitragliatrici}.
    \end{enumerate}
    \subsection{Le info teciche}
    Anche presente una passiva, che fa spownare boccioli nelle vicinanze a caso.
    Da notare com abbia appena un ban rate del 2\%, molto basso se confrontato con il 52\% di vittorie in s11
    Moltiplicati fanno $104=2*52$, che e' un valore sicuramente inutile in quanto calcolato a caso.
    
    \section{I counterpick}
    Generalmente i Counterpick principali di Zyra sono Blitz, perche' non so schivare i suoi benedetti grub, e Bard, perche' esce fuori dal nulla e fa casino.
    Nonostante tecnicamente bard sia piu' efficace, continuo a banare Bliz perche' mi piace di meno giocarci contro.

    \newpage
    \chapter{Seraphine}

    \section{Il Pick}
    \begin{figure}[h]
        \centering
        \includegraphics[width=12cm] {Seraphine.jpg}
        \caption{E-Girl champ}
        \label{fig:Seraphine}
    \end{figure}

    Il campione in figura \ref{fig:Seraphine} e' un altro pick confuso che uso su \emph{LOL}, definito E-Girl champ per ovvie motivazioni.
    Consiglio a chiunque abbia voglia di giocarlo di mutarla appena in partita, diventa /textit{molto} fastidiosa.
    
    Le quattro abilita' di seraphine sono le seguenti:
    \begin{enumerate}
        \item Q, \textbf{High Note}, fa spash damage in un range medio ampio, fa fino al 50\% di \textit{danno} in piu' in base alla salute mancante dell'avversario.
        \item W, \textbf{Surround Sound}, garantisce scudo, movement speed e danno \textcolor{ApBlue}{AP} agli alleati nelle vicinanze, se e' attiva la\nolinebreak
        la passiva, fa anche recuperare vita a tutti quelli in torno in un modo strano che non ho capito benissimo.
        \item E, \textbf{Beat Drop}, manda un inpulso orizzontale con un range sbroccato che fa danno \textcolor{ApBlue}{AP}, slowa per 1 secondo, se il champ e' gia slowato rootta.
        \item R, \textbf{Encore}, altro inpulso orizzontale, charma e slowa chi incontra. Refresha il range ogni volta che passa su un champ alleato o nemico. e, Fa anche abbastanza danno.
    \end{enumerate}
    \subsection{Passiva}
    La passiva di Seraphine e' abbastanza balorda, ogni due abilita' lanciate, la terza viene castata una seconda volta subito dopo senza costare mana.\nolinebreak
    Poi, ogni volta che viene castata un abilita' vengono messe delle "Note" dal comportamento vagamente oscuro che fanno male durante il successivo auto di Seraph.

\end{document}