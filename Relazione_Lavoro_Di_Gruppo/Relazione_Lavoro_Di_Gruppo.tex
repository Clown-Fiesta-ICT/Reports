\documentclass{report}
\usepackage{graphicx}
\usepackage{color}
\definecolor{Green}{RGB}{100, 200, 45}
\definecolor{Blue}{RGB}{75, 0, 200}
\definecolor{Red}{RGB}{255, 0, 0}
\definecolor{Purple}{RGB}{60, 26, 78}

\author{Oglietti Riccardo}
\title{Relazione Lavoro Di Gruppo\\
\large Gruppo 2, Clown-Fiesta-ICT}
\date{\today}

\makeindex

\begin{document}
    \maketitle
    \chapter{Introduzione}
        \section{La Sfida}
            Il progetto che mi accingo a descrivere e' nato come figlio del corso \"Learning By Project\" a opera dei docenti \nolinebreak
             Bardi Laura Silvia e Blachietti Andrea. In esso ci e' stato richiesto di progettare una nuova infrastruttura di \nolinebreak
             rete atta a rimpiazzare l'intera infrastruttura di una scuola superiore Piemontese. La scuola e' divisa in due \nolinebreak
             gruppi di edifici, i quali contengono le sedi di tre differenti indirizzi: Liceo Scientifico delle Scienze \nolinebreak
             Applicate a Indirizzo Informatico, Istituto Tecnico Economico e infine Liceo Linguistico.
            Entrambi i gruppi di edifici sono composti da tre piani, il primo gruppo e' formato da tre edifici, in due \nolinebreak
             dei quali sono concentrate aule, laboratori e locali amministrativi. Nella terza e' poi diposta la \nolinebreak
             e i locali relativi.
            Il seconodo gruppo, non contiguo al primo e' invece formato da due edifici, nel sono dislocate le aule e \nolinebreak
             i laboratori dell'Istituto Tecnico Economico, mentre nel seconodo e' presente la palestra e alcuni edifici\nolinebreak
             amministrativi.
            La progettazione e' stata portata avanti tenendo conto di alcuni punti fondamentali, quali la necessita' di \nolinebreak
             fornire scalabilita' e facilita' di manutenzione da parte dei tecnici presenti all'interno dell'infrastruttura \nolinebreak
             scolastica, e la necessita' di un controllo capillare della sicurezza tramite l'utlizzo di alcuni strumenti \nolinebreak
             fondamentali quali l'infrastruttura MS Active Directory e l'utilizzo di firewall integrati con essa. \newpage
        \section{Il Team}
            La classe e' stata suddivisa in sette differenti gruppi, il nostro gruppo, il numero due, e' formato dai \nolinebreak
             seguenti studenti:
             \begin{itemize}
                 \item \textbf{Catone Mario}
                 \item \textbf{Oglietti Riccardo}
                 \item \textbf{Serena Thomas}
                 \item \textbf{Volgarino Livio}
             \end{itemize}

\end{document}