\documentclass{report}
\usepackage{graphicx}
\usepackage{color}
\definecolor{Green}{RGB}{100, 200, 45}
\definecolor{Blue}{RGB}{75, 0, 200}
\definecolor{Red}{RGB}{255, 0, 0}
\definecolor{Purple}{RGB}{60, 26, 78}
\renewcommand{\chaptername}{Capitolo}

\author{Catone Mario,\\
    Oglietti Riccardo,\\
    Serena Thomas,\\
    Volgarino Livio}
\title{Relazione Lavoro Di Gruppo\\
\large Gruppo 2, Clown-Fiesta-ICT}
\date{\today}

\makeindex

\begin{document}
    \maketitle
    \chapter{Introduzione}
        \section{La Sfida}
            Il progetto che mi accingo a descrivere e' nato come figlio del corso \emph{Learning By Project} a opera dei docenti 
             Bardi Laura Silvia e Blachietti Andrea. In esso ci e' stato richiesto di progettare una nuova infrastruttura di 
             rete atta a rimpiazzare l'intera infrastruttura di una scuola superiore Piemontese. La scuola e' divisa in due 
             gruppi di edifici, i quali contengono le sedi di tre differenti indirizzi: Liceo Scientifico delle Scienze 
             Applicate a Indirizzo Informatico, Istituto Tecnico Economico e infine Liceo Linguistico.

            Entrambi i gruppi di edifici sono composti da tre piani, il primo gruppo e' formato da tre edifici, in due 
             dei quali sono concentrate aule, laboratori e locali amministrativi. Nella terza e' poi diposta la 
             e i locali relativi.

            Il seconodo gruppo, non contiguo al primo e' invece formato da due edifici, nel sono dislocate le aule e 
             i laboratori dell'Istituto Tecnico Economico, mentre nel seconodo e' presente la palestra e alcuni edifici
             amministrativi.

            La progettazione e' stata portata avanti tenendo conto di alcuni punti fondamentali, quali la necessita' di 
             fornire scalabilita' e facilita' di manutenzione da parte dei tecnici presenti all'interno dell'infrastruttura 
             scolastica, e la necessita' di un controllo capillare della sicurezza tramite l'utlizzo di alcuni strumenti 
             fondamentali quali l'infrastruttura MS Active Directory e l'utilizzo di firewall integrati con essa. 

            Un'altra sezione sicuramente fondamentale del progetto e' sicuramente quella legata all'organizzazione di un 
             gruppo di lavoro il piu' efficiente e produttivo possibile, atto a risolvere un problema complesso nella sua 
             struttura. Cio' ci ha permesso di venire in contatto con alcune delle piu' comuni difficolta' del teamwork 
             e di imparare molto su come gestirle, inoltre, trattandosi del primo vero progetto di gruppo per molti di noi 
             l'entusiasmo si e' stato molto fin dalla prima lezione.

        \section{Il Team}
            La classe e' stata suddivisa in sette differenti gruppi, il nostro gruppo, il numero due, e' formato dai 
             seguenti studenti:
             \begin{itemize}
                 \item \textbf{Catone Mario}
                 \item \textbf{Oglietti Riccardo}
                 \item \textbf{Serena Thomas}
                 \item \textbf{Volgarino Livio}
             \end{itemize}
            Durante l'organizzazione del progetto abbiamo optato per suddividere le mansioni principalmente in base agli
             interessi e alle passioni dei singoli componenti, in maniera da rende lo sforzo collettivo quanto piu'
             produttivo possibile. 

            In merito all'organizzazione interna non possiamo dire di aver definito una gerarchia o un vero e proprio 
             "Team Leader", bensi' di esserci organizzati come pari, affidando un carico il piu' possibile uniforme in 
             termini di iportanza e impegno richiesto.

    \chapter{Organizzazione E Strumenti}        
        \section{Comunicazione}
            Come primo passo, durante il primo incontro abbiamo cercato di definire una selezione di strumenti atti a 
             gestire una serie di apetti fondamentali delle dinamiche insite nel teamwork, in primis la Comunicazione
             tra collghi.
            La nostra scelta si e' orientata verso alcuni strumenti chiave, innanzitutto abbiamo optato per la scelta
             della piattaforma \emph{Telegram} per la messaggistica istantanea e le comunicazioni piu' rilevanti grazie
             alle ampie possibilita' di gestione del gruppo, quale la possibilita' di rendere un messaggio prioritario o
             evidenziato. Abbiamo poi deciso di usare lo strumento \emph{GIT} per coordinare la gestione dei documenti
             prodotti da ogni membro del gruppo. Infine, abbiamo optato per lo strumento di conferenza \emph{Discord} per
             gestire gli incontri in remoto al di fuori dell'orario scolastico.
        \section{Strumenti tecnici}
            La selezione degli stumenti tecnici si e' invece rivelata piu' ardua, l'ostacolo piu' grande e' stato coordinare
             le necessita' di ognuno all'interno di un set di strumenti congduo agli obbiettivi del progetto.
            Per quanto riguarda la stesura di relazioni e documenti, abbiamo optato per l'utilizzo del linguaggio \LaTeX,
             esso permette di strutturare documenti estremamente complessi mantenendo una relativa semplicita' di utilizzo.
             Inoltre l'integrazione con lo strumento di sviluppo collaborativo \emph{GIT} e' ottima, e ha permesso di
             ottimizzare lo sforzo comune.
            %Per facilitare la nostra organizzazione interna abbiamo da subito deciso di assegnare un colore a ogni membro, 
             %tramite questo espediente ci e' stato piu' facile organizzare il lavoro e i compiti di ognuno in maniera 
             %chiara e visuale.

\end{document}