\documentclass{report}
\usepackage{graphicx}
\usepackage{color}
\definecolor{Green}{RGB}{100, 200, 45}
\definecolor{Blue}{RGB}{60, 0, 250}
\definecolor{Red}{RGB}{255, 0, 0}
\definecolor{Purple}{RGB}{60, 26, 78}
\renewcommand{\chaptername}{Capitolo}
\renewcommand{\contentsname}{Indice}

\author{Catone Mario,\\
    Oglietti Riccardo,\\
    Serena Thomas,\\
    Volgarino Livio}
\title{Relazione Lavoro Di Gruppo\\
\large Gruppo 2, Clown-Fiesta-ICT}
\date{\today}

\makeindex

\begin{document}
    \maketitle
    \tableofcontents
    \chapter{Introduzione}
        \section{La Sfida}
            Il progetto che mi accingo a descrivere e' nato come figlio del corso \emph{Learning By Project} a opera dei docenti 
             Bardi Laura Silvia e Blachietti Andrea. In esso ci e' stato richiesto di progettare una nuova infrastruttura di 
             rete atta a rimpiazzare l'intera infrastruttura di una scuola superiore Piemontese. La scuola e' divisa in due 
             gruppi di edifici, i quali contengono le sedi di tre differenti indirizzi: \textbf{Liceo Scientifico delle Scienze 
             Applicate a Indirizzo Informatico}, \textbf{Istituto Tecnico Economico} e infine \textbf{Liceo Linguistico}.

            Entrambi i gruppi di edifici sono composti da tre piani, il primo gruppo e' formato da tre edifici, in due 
             dei quali sono concentrate aule, laboratori e locali amministrativi. Nella terza e' poi disposta la palestra
             e relativi locali.

            Il seconodo gruppo, non contiguo al primo e' invece formato da due edifici, nel sono dislocate le aule e 
             i laboratori dell'Istituto Tecnico Economico, mentre nel seconodo e' presente la palestra e alcuni edifici
             amministrativi.

            La progettazione e' stata portata avanti tenendo conto di alcuni punti fondamentali, quali la necessita' di 
             fornire scalabilita' e facilita' di manutenzione da parte dei tecnici presenti all'interno dell'infrastruttura 
             scolastica, e la necessita' di un controllo capillare della sicurezza tramite l'utlizzo di alcuni strumenti 
             fondamentali quali l'infrastruttura MS Active Directory e l'utilizzo di firewall integrati con essa. 

            Un'altra sezione sicuramente fondamentale del progetto e' sicuramente quella legata all'organizzazione di un 
             gruppo di lavoro il piu' efficiente e produttivo possibile, atto a risolvere un problema complesso nella sua 
             struttura. Cio' ci ha permesso di venire in contatto con alcune delle piu' comuni difficolta' del teamwork 
             e di imparare molto su come gestirle, inoltre, trattandosi del primo vero progetto di gruppo per molti di noi 
             l'entusiasmo si e' stato molto fin dalla prima lezione.

        \section{Il Team}
            La classe e' stata suddivisa in sette differenti gruppi, il nostro gruppo, il numero due, e' formato dai 
             seguenti studenti:
             \begin{itemize}
                 \item \textbf{Catone Mario}
                 \item \textbf{Oglietti Riccardo}
                 \item \textbf{Serena Thomas}
                 \item \textbf{Volgarino Livio}
             \end{itemize}
            Durante l'organizzazione del progetto abbiamo optato per suddividere le mansioni principalmente in base agli
             interessi e alle passioni dei singoli componenti, in maniera da rende lo sforzo collettivo quanto piu'
             produttivo possibile. 

            In merito all'organizzazione interna non possiamo dire di aver definito una gerarchia o un vero e proprio 
             "Team Leader", bensi' di esserci organizzati come pari, affidando un carico il piu' possibile uniforme in 
             termini di iportanza e impegno richiesto.

    \chapter{Organizzazione E Strumenti}        
        \section{Comunicazione}
            Come primo passo, durante il primo incontro abbiamo cercato di definire una selezione di strumenti atti a 
             gestire una serie di apetti fondamentali delle dinamiche insite nel teamwork, in primis la Comunicazione
             tra collghi.
            La nostra scelta si e' orientata verso alcuni strumenti chiave, innanzitutto abbiamo optato per la scelta
             della piattaforma \emph{Telegram} per la messaggistica istantanea e le comunicazioni piu' rilevanti grazie
             alle ampie possibilita' di gestione del gruppo, quale la possibilita' di rendere un messaggio prioritario o
             evidenziato. Abbiamo poi deciso di usare lo strumento \emph{GIT} per coordinare la gestione dei documenti
             prodotti da ogni membro del gruppo. Infine, abbiamo optato per lo strumento di conferenza \emph{Discord} per
             gestire gli incontri in remoto al di fuori dell'orario scolastico.
        \section{Strumenti Tecnici}
            La selezione degli stumenti tecnici si e' invece rivelata piu' ardua, l'ostacolo piu' grande e' stato coordinare
             le necessita' di ognuno all'interno di un set di strumenti congduo agli obbiettivi del progetto.
            Per quanto riguarda la stesura di relazioni e documenti, abbiamo optato per l'utilizzo del linguaggio \LaTeX,
             esso permette di strutturare documenti estremamente complessi mantenendo una relativa semplicita' di utilizzo.
             Inoltre l'integrazione con lo strumento di sviluppo collaborativo \emph{git} e' ottima, e ha permesso di
             ottimizzare lo sforzo comune. Da notare poi la possibilita' di gestire i documenti tramite il comodo sistema
             basato su commit sul quale si basa \emph{git}, e' stato largamente piu' semplice gestire le revisioni
             collaborative per qualsivoglia documento, e la scrittura collaborativa, indispensabile per la redazione delle
             relazioni.
            Per cio' che concerne invece la creazione di diagrammi di rete, e' stato scelto lo strumento gratuito
             \emph{Draw.io}, si tratta di una Webapp atta a creare rappresentazioni grafiche di qualsiasi genere,
             il tool e' stato scelto per via della sua semplicita' d'uso e della possibilita' di esportare liberamente
             i documenti creati.
        \section{Definizione Dei Ruoli}
            Infine, abbiamo deciso di operare alcune scelte organizzative a nostro avviso opportune per poter gestire il
             progetto nella sua interezza. Abbiamo optato per un organizzazione basata su una suddivisione in ruoli, con
             mansioni specifiche e meidamente lunghe, per poi coordinare gli sforzi durante meeting con catenza settimanale.
            Per facilitare la nostra organizzazione interna abbiamo da subito deciso di assegnare un colore a ogni membro,
             tramite questo espediente ci e' stato piu' facile organizzare il lavoro e i compiti di ognuno in maniera chiara
             e visuale. 
            La suddivisione e' quindi risultata come segue: 
            \begin{center}
                \begin{tabular}{ |c|c| }
                    \hline
                    Nome Componente & Compito Assegnato \\
                    \hline \hline
                    \textcolor{Blue}{Catone Mario} & \textcolor{Blue}{Active Directory} \\
                    \hline
                    \textcolor{Purple}{Oglietti Riccardo} & \textcolor{Purple}{Reportistica e presentazione} \\
                    \hline
                    \textcolor{Red}{Serena Thomas} & \textcolor{Red}{Firewall e sicurezza} \\
                    \hline
                    \textcolor{Green}{Volgarino Livio} & \textcolor{Green}{Topoogia di rete e routing} \\
                    \hline
                \end{tabular}
            \end{center}
            Questa particolare suddivisione dei ruoli e' nata dopo un confronto sulle nostre tematiche di interesse, e sui
             nostri punti di forza principali. Per citare un esempio, \textcolor{Blue}{Catone Mario} a scelto la sua mansione
             dopo aver seguito con interesse il corso proposto per l'amministrazione di sistemi basati su \emph{Windows e MS
             Active Directory} di Cristante Fabrizio e aver riscontrato un grande interesse sulla tematica. Invece,
             \textcolor{Red}{Serena Thomas}, rimasto molto colpito dalle implicazioni legate alla sicurezza studiate e
             sperimentate durante il corso di \emph{Firewall} a opera di Vedovato Alberto, ha deciso di intraprendere
             questa mansione.

    \chapter{Diagramma Di Rete E Topologia}
        \author{Volgarino Livio}
        \section{La Struttura In Generale}
        Come precedentemente specificato, la struttura di rete e' stata ideata principalmente da Volgarino Livio, ed e
         implementata dedicando particolare attenzione ad alcuni aspetti giudicati pilastri fondamentali del progetto:
         \begin{itemize}
             \item La separazione delle reti, e' stato scelto di progettare due reti separate e largamente indipendenti
              in modo da rendere piu' agevole l'installazione e il mantenimento, inoltre un eventuale fallimento
              dell'infrastruttura in uno dei due edifici non penalizzerebbe la didattica all'interno del secondo plesso.
             \item L'utilizzo di indirizzamento statico per i dispositivi fissi, propri dell'istituto (quali per esempio
              postazioni all'interno dei laboratori informatici in ambo i plessi), estremamente utile per mantenere un'
              estrema semplicita' di gestione e di troubleshooting
             \item Filtraggio di tutto il traffico da parte di router/firewall che agiscono da endpoint per ognuno dei due
              plessi, cio' permette di innalzare lo standard di sicurezza, in quanto tutto il traffico nella sua interezza
              viene analizzato.
         \end{itemize}
        Andremo ora a spiegare nel dettaglio l'architettura, aiutati dal diagramma di rete redatto tramite il precedentemente
         citato strumento di creazione grafici e diagrammi \emph{Draw.io}.
        \section{Il Diagramma}
        La rete nel suo complesso puo' essere riassunta tramite l'utilizzo di questo diagramma:
    \chapter{Sicurezza E Firewall}
        \author{Serena Thomas}
    \chapter{Microsoft Active Directory E Windows Licensing}
        \author{Catone Mario}
        \section{Domain Controller}
        L'ambiente \emph{MS Active Directory} e' il cuore pulsante dell'intera infrastruttura di rete, in quanto tutta 
         la gestione dell'autenticazione, e dei relativi permessi dedicati a ogni singolo utente, dagli amministratori
         di sistema agli studenti, viene gestita tramite questo sitema.
        Abbiamo optato per una configurazione basata su due domain controller, uno per plesso, in modo da garantire la
         continuita' del servizio anche in caso di problemi hardware.
        Le macchine sono strutturate come segue:
        \begin{itemize}
        \item DC01:
            \begin{itemize}
            \item Dell Smart Value PowerEdge T140 con 1TB HDD, 2c/2t e 8GB di RAM ECC. \textcolor{Red}{$\rightarrow$ da rivedere con Xeon}
            \item Windows Server 2019 Standard Edition Desktop Experience. %richiesta revisione
            \end{itemize}
        \item DC02: 
            \begin{itemize}
            \item Dell Smart Value PowerEdge T140 con 1TB HDD, 2c/2t e 8GB di RAM ECC. \textcolor{Red}{$\rightarrow$ da rivedere con Xeon}
            \item Windows Server 2019 Standard Edition. %richiesta revisione
            \end{itemize}
        \end{itemize}
        Le macchine sopracitate, seppur dotate di perfomance modeste in relazione agli standard odierni, offrono un ottimo 
         rapporto qualita' prezzo, inoltre, entrambi i \emph{Domain Controller} non si troveranno a gestire mansioni 
         particolarmente intensive dal punto di vista computazionale.
        Le funzioni principali di queste due macchine sono sintetizzabili come segue:
        \begin{enumerate}
            \item Server di autenticazione per l'intera infrastruttura.
            \item Mail server per docenti e personale amministrativo.
            \item Gestione Shared Directory.
            \item Gestione Roaming Profiles.
        \end{enumerate}
            \subsection{Server Di Autenticazione}
            Il ruolo di \textbf{Server Di Autenticazione} da parte del \emph{Domain Controller} e' di vitale importanza
             per il corretto funzionamento dell'infrastruttura nel suo complesso. Esso permette di gestire su base di 
             gruppi o di singoli utenti molti aspetti dell'esperienza utente, che possono variare dalla configurazione
             dell'ambiente desktop e del menu' start, a eventuali permessi di scrittura, lettura e amministrazione di
             directory condivise, fino alla possibilita' di amministrare il sistema nella sua interezza.
            
            In tutto cio' le \emph{Organizational Unit} anche dette \textbf{OU} assumono un ruolo fondamentale. Si 
             tratta di "Cartelle", atte a organizzare gli utenti in sottogruppi per renderne la gestione piu' agevole,
             nel nostro caso, abbiamo ritenuto opportuno suddividere gli utenti come segue:

            \begin{itemize}
                \item Liceo Scientifico delle Scienze Applicate a Indirizzo Informatico:
                    \begin{itemize}
                        \item OU Studenti Liceo
                        \item OU Docenti Liceo
                        \item OU Personale Scolastico
                    \end{itemize}
                \item Liceo Linguistico:
                    \begin{itemize}
                        \item OU Studenti Liceo
                        \item OU Docenti Liceo
                        \item OU Personale Scolastico
                    \end{itemize}
                \item Istituto Tecnico Economico:
                    \begin{itemize}
                        \item OU Studenti Economico
                        \item OU Docenti Economico
                        \item OU Personale Scolastico
                    \end{itemize}
            \end{itemize}
            I motivi di questa particolare suddivisione sono presto detti: %richiesta revisione
            \subsection{Server Mail}
            \subsection{Gestione e Immagazzinamento Shared Directory}
            \subsection{Gestione Roaming Profiles e Persistenza Profili Utenti}
    \chapter{Considerazioni Finali}
            

\end{document}